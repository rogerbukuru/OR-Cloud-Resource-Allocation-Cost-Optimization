
\documentclass{beamer}
\usepackage{amsmath}
\usepackage{graphicx}
\usepackage{xcolor}

\title{Optimization of Energy Efficiency and Execution Time in Cloud Computing}
\subtitle{Using MILP, Simulated Annealing, Genetic Algorithms, and Multi-Objective Goal Programming}
\author{Tom Jardin}
\date{}

\begin{document}

\frame{\titlepage}

\begin{frame}{Introduction}
    \begin{itemize}
        \item As cloud computing grows, so does energy consumption in data centers, leading to increased operational costs and CO2 emissions.
        \item This project explores how optimization techniques can enhance energy efficiency and reduce execution time in cloud environments.
    \end{itemize}
\end{frame}

\begin{frame}{Problem Statement}
    \begin{itemize}
        \item \textbf{Objective}: To minimize energy consumption and execution time in a cloud computing environment.
        \item \textbf{Key Features}: CPU usage, memory usage, network traffic, power consumption, execution time, task type, task priority, task status.
    \end{itemize}
\end{frame}

\begin{frame}{Mixed Integer Linear Programming (MILP)}
    \textbf{Objective}: Minimize the total cost of energy consumption and execution time.
    \begin{align*}
        \text{Minimize } Z &= \alpha \times \text{Energy Consumption} + \beta \times \text{Execution Time}
    \end{align*}
    \textbf{Subject to:}
    \begin{align*}
        &\text{Resource Constraints: } \sum_{i} r_{ij} \times x_{it} \leq R_{j}^{\text{max}}, \quad \forall j, t \\
        &\text{Task Completion Constraints: } \sum_{t} x_{it} = 1, \quad \forall i \\
        &\text{Binary Constraints on decision variables.}
    \end{align*}
\end{frame}

\begin{frame}{Simulated Annealing (SA)}
    \textbf{Objective}: Iteratively minimize energy consumption and execution time by exploring different configurations.
    \begin{itemize}
        \item Start with an initial solution \(S\) and initial temperature \(T\).
        \item For each iteration, generate a new solution \(S'\) in the neighborhood of \(S\).
        \item Compute the change in cost \(\Delta E = E(S') - E(S)\).
        \item If \(\Delta E < 0\), accept \(S'\). Else, accept \(S'\) with probability \(\exp(-\Delta E / T)\).
        \item Decrease the temperature \(T\) and repeat until convergence.
    \end{itemize}
\end{frame}

\begin{frame}{Genetic Algorithms (GA)}
    \textbf{Objective}: Evolve task allocation strategies to minimize energy consumption and optimize execution time.
    \begin{itemize}
        \item Initialize a population of chromosomes (task allocations).
        \item Evaluate fitness: \(f(\text{chromosome}) = w_1 \times \text{Energy Consumption} + w_2 \times \text{Execution Time}\).
        \item Selection: Choose the fittest chromosomes.
        \item Crossover: Create offspring by combining pairs of chromosomes.
        \item Mutation: Introduce small changes to chromosomes.
        \item Repeat until convergence or max iterations.
    \end{itemize}
\end{frame}

\begin{frame}{Multi-Objective Goal Programming (MOGP)}
    \textbf{Objective}: Balance multiple goals like minimizing energy consumption and execution time.
    \begin{itemize}
        \item Define Goals: \(G_1\) - Minimize Energy, \(G_2\) - Minimize Execution Time.
        \item Formulate deviation variables (\(d^+\) and \(d^-\)) for each goal.
        \item Minimize the weighted sum of deviations: \(\text{Minimize } Z = w_1 \times d_1^+ + w_2 \times d_2^-\).
        \item Subject to: Goal constraints, resource constraints.
    \end{itemize}
\end{frame}

\begin{frame}{Implementation Strategy}
    \begin{itemize}
        \item Implement the problem using R, with focus on data preprocessing, algorithm application, and evaluation metrics.
        \item \textbf{Data Preparation}: Extract and normalize key features from the dataset.
        \item \textbf{Algorithm Application}: Apply MILP, SA, GA, and MOGP to optimize the objectives.
    \end{itemize}
\end{frame}

\begin{frame}{Expected Outcomes}
    \begin{itemize}
        \item Anticipated results include improved energy efficiency, reduced execution time, and insights into trade-offs between objectives.
        \item Evaluation will be based on how well the optimization techniques balance energy consumption and execution time.
    \end{itemize}
\end{frame}

\end{document}
